    Slide 1: Intro 
        - we studied blah



So gravitational waves were first predicts in 1893 and then another time in 1905, by people inspired to adapt Maxwells equations to gravity. However, they really came to the forefront of discussion when Einstein introduced the General Theory of relativity. As these were just such perfect equations as Maxwells were and they also seemed to under certain assumptions predict Gravitational waves. Now Einstein never fully believed in them himself, and this would spark major debates over the next 50-60 years, and this history is quite fascinating with many big names, but we would be here for a very long time, and I was told that you guys won't be interested in the history. So we will skip to when the first discovery of Gravitational waves in 1974.  



    
    Slide 2: Newton
        -while know apple tree story is not true, newton did discover gravity
        -failed to predict gravitational waves but did create law of gravitation in 1687
        -law is gravitational force equals gravitational constant time (mass 1 * mass 2) divided the distance between the two objects center of mass squared
    
    Slide 3: Oliver Heaviside & Henry Poincare
        -Oliver Heaviside 1893
        Our story starts with Oliver Heaviside who in 1893 was studying the relationship between Maxwell's equations and gravity. Heaviside performed a step by step recreation of Maxwell's equations on gravity and reached wave equations for gravity. He uses these equations to calculate the relationship of gravity between the Earth and Sun and concludes that there must be numerous orbital perturbations, and thus, is the first one to predict the possibility of Gravitational Waves.
        -Henry Poincare 1905
        In 1905, Henri Poincaré proposed gravitational waves, emanating from a body and propagating at the speed of light, as being required by the Lorentz transformations and suggested that, in analogy to an accelerating electrical charge producing electromagnetic waves, accelerated masses in a relativistic field theory of gravity should produce gravitational waves.
    
    
    Slide 4-6: Einstein
        Then in comes Albert Einstein, and as we all know, he introduces the General Theory of Relativity in 1915. However as for the matter that Poincare predicted he was not entirely convinced at the time, even writing to his colleague Karl Scharzchild, “Since then [November 14] I have handled Newton’s case differently, of course, according to the final theory [the theory of General Relativity]. Thus there are no gravitational waves analogous to light waves. This probably is also related to the one-sidedness of the sign of the scalar T, incidentally [this implies the nonexistence of a “gravitational dipole”].
    
        However, Einstein wasn't fully convinced either way, and he himself with a few assumptions produced 3 possible types of gravitational waves, dubbed longitudinal–longitudinal, transverse–longitudinal, and transverse–transverse.
    
    Slide 7: Eddington
        Here incomes Arthur Eddington, a great astrophysicist and mathematician, and was one of the first people to rigorously test Einsteins equations on predicting gravity.   
    
        He disproves the first 2 types of waves that Einstein proposed, and proves that if the 3rd type exists it will travel at the speed of light.
    
    Slide 8: 
        A few years later after coming to America, while working with an assistant Nathan Rosen, Einstein where they would conclude that there cannot be a complete theory of relativity in which gravitational waves exist. 
        He wrote a paper addressing gravitational waves called it Are there any graviational waves, in which concludes "They do not exist", and sent it to the Physics Review to be published. However the editor sent this paper to be peer reviewed by Robertson, who made several negative comments on Einstein's paper, throwing him into a fit of rage and not publishing paper. 
        Later Infeld will show Einstein the error in calculations and Einsteins stance would change to "I don't know" and they would publish the paper with a complete 180 degree conclusion that same year. 
    
    Slide 9:
        Pirani would remedy many confusions and propose a reasonable way to measure gravitational waves in 1956. 
    Slide 10:
        However, his work was overshadowed by the debate on whether or not gravitational waves could transmit energy, which would be solved by a hypothetical experiment from Richard Feynman which basically states that if one takes a rod with beads then the effect of a passing gravitational wave would be to move the beads along the rod; friction would then produce heat, implying that the passing wave had done work, therefore gravitational waves would produce energy.
        Weber Bars, the first gravitational wave detectors built, results were criticized.    
    
    Slide 11:
        First discovery:
        This finally leads to first actual discovery of gravitational waves in 1974 from the observation of orbital decay of the Hulse-Taylor binary pulsar, a high spped spinning neutron star with a binary companion, named after the 2 people Russell A Hulse and Joseph Hooton Taylor who were studying this and discovered gravitational waves, later wining the Nobel Prize for their discovery.
    
    Slide 5: GWs Indie
        So after all this talk. What are gravitational waves?
        -gws= ripple in space time caused by massive accelerating objects colliding - black holes, nuetron stars, supernovas
       
    DIGRAM OF a LASER INTERFEROMETER

    
    
    
    Slide 6: LIGO Arnold
        -Laser Interferometer Gravitational-Wave Observeratory
        -tries to detect GWs caused by cosmic events
        - in Hanford Washington and livingston louisiana
        -started in 1992
        -in operation 2002
        -2011-2014 - fixed and updated
        -2015 sept & dec first GW detections
        -still running
        - legs 4 km long (2.5 miles)
        -laser goes from one end to other if does not hit exact point where supposed to hit something disturbed it
        
        
    
    Slide 7: LISA Indie
        -Laser Interferometer Space Antenna
        -first sketchs in 1980s
        -measure frequencies from 0.1 mHz to 100 mHz
        -launch date in mid 2030s
        -ESA has lead but NASA helping
        -2011 NASA left 2018 NASA rejoined
       
    Slide : Lisa Arnold
    -two free falling test masses in center if move mean there is acceleration noise hopefully from gravity
    -3 pieces 5*10^6 apart in equilateral triangle rotating clockwise 
    -20 degrees behind earth
    
    
    Slide 8: LPF indie
        -first step towards LISA basically test run
        -launched 3 december 2015
        -1.5 million km from earth
        -march 2016 opperational
        -did better than expected
        -much more sensitive
    
    Slide 9: Solar Wind arnold
        - Solar wind results from increasing temperature the sun's corona till a certain threshold at which the energy from the height outweighs the sun's power gravity, causing a coronal mass ejection  
        -These ejections can be shot out at a speed of up to 800 km/s and have a devastating effect on satellites
        - For example when your GPS says your a building away from where you really are you can probably blame in solar wind
    
    Slide 10: ACE-Advanced Composition Explorer
        - The ACE satellite is flown also circling Lagrange Point 1
        - The ACE satellite is used to measure Solar Wind data including, all these statistics that you don't need to completely get: proton speed, density, proton to alpha particle ratio, xyz components of proton velocity and also UTC time
        
    Slide 11: Methods
        - talk about diagram
    
    Slide 12: gap fill
        - talk about how gap fill works
    
    Slide 13: Equations
        -talk about what equations used and why
        -so these equations convert all the data collected from ACE and convert it into Force data modeling the LPF satellite allowing direct comparison between the 2 sets of data.
        
    
    Slide 14: Time series
    Slide 15: FFT
    add signal to all 3
    Slide 16: Bode Magnitude
    look up bode stuff
    Slide 17: Bode Phase
    Slide 18: Coherence
    rainbow coherence plot add
    Slide 19: Noise Cancellation if we get it to work
    Slide 20: Conclusion
        - we found that spurious solar wind does not have an effect on the acceleration noise of LPF data
    Slide 21: sources
    
    